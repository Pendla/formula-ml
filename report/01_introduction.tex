\chapter{Introduction}
% Introduce why machine learning is important and why it has received so much popularity if the last few years.
Artificial intelligence (AI) is an area that currently is receiving a lot of attention at the moment.  and an example of this is the current development of autonomous vehicles.    

        
\section{Background}



\iffalse
Artificial intelligence (AI) has been a large area of research in computer science for a long time. In 1997 AI research had a major breakthrough, when an AI beat the reigning world champion in chess for the first time ever. Following this breakthrough were AIs such as IBMs "Watson" which beat the two leading world champions in Jeopardy, personal assistants in smartphones such as Apples "Siri" and just recently Google presented an AI that beat a top tier player in the classical game Go.

AI keeps proving itself to be an important concept in computer science and is thus one that is of great interest to computer scientists. There are many ways in which to approach AI development and in most real world scenarios a combination between several approaches will most likely be the optimal solution. However one area that has received much attention lately is the combination between machine learning and neural networks. This project will aim to research this concept further in order to broaden our knowledge about this area and to examine whether or not a computer can learn complicated behaviours, not only function approximations through machine learning.

Machine learning in its simplest form, being used to solve a relatively simple problem is not very complicated. However it becomes vastly more complicated when applied to a more complex problem, where a simple rule based solution does not really suffice in order to solve the problem well enough. In order to effectively research machine learning this project will take advantage of this fact. The project will focus around the problem of taking a Formula1 (F1) car as fast as possible around any given F1 circuit, a problem that is very flexible in terms of complexity. This will allow the project to start of with the very basics behind machine learning then by slowly increasing the complexity of the problem we can closely examine the more complex concepts behind machine learning.

The potential applications that AI and machine learning can be used within are as one might imagine endless. In this specific project, it's interesting to consider the gaming industry and racing industries for example. Could such an AI be used in racing games? Maybe even for game AIs in general? Could it potentially be of help to the F1 drivers during practice or even for the team engineers? 



% Introduce machine learning and racing (as a sport). Explain how these are related and what is interesting with this relation.
Many problems in computer science and engineering are hard to solve programatically. It may be difficult or costly to develop accurate models and processes. One attempt to solve some of these problems is to use various machine learning techniques. 

If some of the task to model and develop procedures is left to a machine learning process, hopefully less domain knowledge is required by the application developer. 

Example: robot arm movement?

In a similar way, machine learning may be utilised solved many other tasks in the future.

In order to properly make well use of machine learning, one need to understand well how the machine learning process works, what the possibilities are and the limitations on what role it can have.
\fi



\section{Problem}
% Each of the sections needs to be broadened, but it is a start
Driving a car and racing is a complex behaviour that both contain strategic behaviours and driving the car to it's limit to get the best result as possible. The optimal speed and line of the car is different for each car and depends on grip and acceleration, which both depends on various of variables each. It also depends on how the track/environment looks further down. If there is a curve followed be a straight line, the fastest way is to carry as much speed as possible out from the curve. A curve followed by another curve, would instead have to be calculated to have the best position to the next curve. These are just examples of how complex behaviour a race driver do need to have to drive the car to the limit.

The controls of a car are not discrete values, as pedals and steering are continuous values within an interval. This creates a boundary (citation needed) where algorithms and methods that utilises decision trees and discrete states non applicable to the racing domain. 

There are no racing data available from previous races for the research to apply to any machine learning algorithm. If there would have been any data that were available, the virtual domain would have to be set up to match that setup of track, car and environment when the data was performed. This, just as the continuous control values, create boundaries of what methods that can be used (citation needed). There is no sample data to use in the algorithms. 

\section{Purpose}
This paper will focus on how well a machine learning algorithm can drive a car around a track, and optimise the way of driving to make it drive as fast as possible. The purpose of the paper is to see if it is possible to achieve such a complex behaviour. What types of hazels occur when implementing and running the algorithms will be brought up in the paper, as well as thoughts and results of different experiments and tweaking of the algorithm.

%Should limitation be a subsection instead? separate section. 
The experiments and implementation in this paper will be limited to different aspects of racing. There will be no focus regarding racing towards other cars, only focusing on optimising single lap times. The car will also only do one lap, and not try to optimise for a larger set of laps.

The dynamics of a car is very complex and hard to model. To calculate the maximum turning radius in a given moment are determined by multiple variables. This paper will simplify this model and %GO on 
\section{Limitation}

\section{Literary Review}
% Instruction from https://www.reading.ac.uk/internal/studyadvice/StudyResources/Essays/sta-structuringreport.aspx#literature
% The Literature Review… surveys publications (books, journals and sometimes conference papers) on work that has already been done on the topic of your report. It should only include studies that have direct relevance to your research.
% Introduce your review by explaining how you went about finding your materials, and any clear trends in research that have emerged. Group your texts in themes. Write about each theme as a separate section, giving a critical summary of each piece of work, and showing its relevance to your research. Conclude with how the review has informed your research (things you'll be building on, gaps you'll be filling etc).
% Further reading: http://www.rlf.org.uk/resource/what-is-a-literature-review/
