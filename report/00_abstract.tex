\begin{abstract} %Needs rework and more describing, selling text.
This paper presents a study of how a simulated race car would behave by learning how to drive a predefined track by reinforcement learning. Racing is a complex and continuous domain, with a lot of variables that are useful in different scenarios, and some variables that may be in excess. The paper describes an evolutionary algorithms NEAT that have been adapted to fit the race simulator and domain. It also focuses on search space reduction to speed up learning and computation.
\end{abstract}


% ------------------------------------------------------------------------------------

% Introduction
% Method
% Result
% Discussion

% ## Theory

% Racing theory / Properties we want the AI to have / Evaluation (Method)
% Simulation

% Relevant ML techniques (In Method)
%     Reinforcement learning
%     Genetic algorithms
%     Search Space reduction

% Other ML techniques
%   Supervised
%   Unsupervised
%   Markov decision problem
%   Linear programming
%   Dynamic programming
 
 

% ## What have we done
% Methods
%     Exhaustive search
%     NEAT (FS-NEAT)

% Indata
%     Grid data as input
%     Curve data as input
%         Checkpoints
%         Curve
%         Sum of absolutes
%         ...
        
% Evaluation tools
%     Distance driven by car on track
%     Neural Network visualisation.
    
% Saving/Loading networks

% Ideas
%     Stuck in local minima 
%     Heurustics
%         Benefit som types of crashed before others
%         Benefit recing line properties

% (Architecture/design of application)

% (Graphics)


% Result
%     Stick to mid line
%     Behaviour
%     Difference in training times and result for different settings