% Bakgrund
% Bakgrund ska innehålla en motivering till varför det valda ämnet är intressant ur akademisk synvinkel och/eller ur tekniskt perspektiv eller i förekommande fall ur kundens/uppdragsgivarens perspektiv. I vissa fall ska den här rubriken inkludera en kort historik över ämnet. Efter att ha läst bakgrunden ska alla läsare förstå varför ämnet är relevant. Följande frågeställningar bör vara aktuella:
% Vad är ämnet/problemet som ska undersökas? Varför har ämnet/problemet uppkommit? Varför är det ett relevant eller intressant ämne/problem? För vem? Kan det specifika ämnet/problemet relateras till en mer generell diskussion?

\chapter{Background}
The field of autonomus vehicles is growing. This year Volvo is releasing 100 self driving cars on the roads around Gothenburg, Sweden. These cars are not self driving in that matter that the learn to drive themselves by try and error, as these cars are way to expensive to have them drive without any guidance, not to speak about having them in traffic without any instructions how to behave. 

In other areas were testing also is limited, IE. Formula 1 and other motor sports have restrictions on how much testing each car and team can utilise. In all these cases simulations are necessary to evaluate as much as possible before each test, to make use of the scarce resource of testing in the best way possible. One thing that the driver needs to do is figure out what race line is the optimal for the current car and weather conditions, and if a computer would be able to give this information to the driver beforehand, the driver could focus on other things during practice and testing. An already determined optimal race line would also be of a helpful tool to analyse the driving and see how much each driver deviates from the optimal race line.

The goal of this project is to see if it is possible to help the driver finding the optimal race line for a track before the driver would even have to drive around the track. To make the car/computer learn by itself 



\section{Bra frågor att ställa sig}

\begin{itemize}
    \item Vad är ämnet/problemet som ska undersökas?  
    Creating an autonoumus formula one car using machine learning.
        Köra snabbt, inte köra ur
        
    \item Varför har ämnet/problemet uppkommit? 
    Machinelearning! Går att lösa enkelt, går att köra avancerat.
    Det är svårt att hitta enkla regelbaserade beteenden för många problem, eller traditionella algoritmer. Det Formel1 specifika delarna av projektet är en bra lekstuga som ger möjlighet att gå från ett grundläggande problem till ett mer avancerat.
    Skulle kanske kunna lära sig något intressant för racing spel, men det är inte där det fokus ligger? Skulle detta kunna vara till hjälp för generell spel ai?
    
    \item Varför är det ett relevant eller intressant ämne/problem? 
    (Formel 1), autonoma fordon, machinelearning i allmänhet.
    
    \item För vem? 
    För oss. För intresserade av machinelearning. 
    Machinelearning: Möjliggör smartare program som hjälper individen mer. Racingbranchen?
    
    \item Kan det specifika ämnet/problemet relateras till en mer generell diskussion? 
    Se fråga "varför är det relevant"
    Can a computer learn complicated behaviours, not only function approximations, through machine learning?
    
    \item Discussion on Effect goals. 
    Learn machine learning. this formula 1 project is a sandbox.
    
\end{itemize}

\begin{itemize}
  \item Motivate the purpose of the report.
  \item What is the source of the problem?
  \item Why is it interesting and/or relevant?
  \item Usually ~1 page or less.
  \item Discussion on Effect goals.
\end{itemize}

