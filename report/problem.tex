% Problem/Uppgift
% Det här avsnittet är ofta den viktigaste delen av planeringsrapporten (och av den slutgiltiga rapporten). Den syftar till att identifiera frågan/frågorna som ska tas upp i projektet. Det är viktigt att gruppen gör en problem(uppgifts)analys även om det i projektförslaget redan finns ett problem (en uppgift) specificerat. Anledningen till detta är att det riktiga primära problemet ofta skiljer sig från det i början av uppdragsgivaren/förslagsställaren/kunden föreslagna. Problemanalysen syftar också till att bryta ner problemet/uppgiften i mindre och mer detaljerade delproblem/deluppgifter, vilket också leder till formulering av delsyften. Genom att göra detta får studenterna mycket bättre förståelse för de olika aspekterna av problemet/uppgiften. Utan den här informationen är det omöjligt att identifiera vilken information som behövs, vilka informationskällor som behövs,  och lämpliga tillvägagångssätt.

% En bra problemanalys som identifierar delproblem/deluppgifter och delsyften vilar i många fall på användning av teorier och modeller från litteraturen. En litteraturgenomgång bör därför genomföras tidigt i processen.
\chapter{Problem Description}

  * Analyze what issues and questions that will be covered in the report.
  * Identify the different parts of the problem. Motivation of milestones.
  
  Method/Algorithms
    Requirements for different variants. Elaborate as mucha as we can at this point. Continue in report when we know more and have experimental results
    Time complexity
    Feasible in practise
    without training examples...
    
    Training/Search
      How do we find solutions that work well? Or improve?
      That wanted features are developed? Features are not forgotten?
      Over trained?
    
    
  Simulation
    Accuracy
    Realism
    What properties of the real world are relevant for the problem
    How important are the different properties for the behaviour
  
  