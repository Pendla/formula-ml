\chapter{Introduction}

Racing requires several aspects of tactical behaviour. The driver might need to slow down before a curve in order to turn enough. The driver also need to position the car well before curves in order to have a large enough turning radius, to enable higher speeds, and still not drive a too long distance.

This purpose of this paper is to explore the usage and role of the machine learning technique NEAT in the context of solving the racing problem, using little domain knowledge. Furthermore it will discuss machine learning in general for problems with similar types of characteristics.

The paper will focus on some of the key behaviours of racing and not discuss all aspects that may be relevant for a realistic setting. It will also only discuss racing for one single car, and not the aspects of racing where several cars compete at the same time. 


\section{Background} %Copied from planning
Artificial intelligence (AI) has been a large area of research in computer science for a long time. In 1997 AI research had a major breakthrough, when an AI beat the reigning world champion in chess for the first time ever. Following this breakthrough were AIs such as IBMs "Watson" which beat the two leading world champions in Jeopardy, personal assistants in smartphones such as Apples "Siri" and just recently Google presented an AI that beat a top tier player in the classical game Go.

%TODO change structure to below
%and Google's AI that have just recently beaten a top tier player in the classical game Go.

AI keeps proving itself to be an important concept in computer science and is thus one that is of great interest to computer scientists. There are many ways in which to approach AI development and in most real world scenarios a combination between several approaches will most likely be the optimal solution. However one area that has received much attention lately is the combination between machine learning and neural networks. This project will aim to research this concept further in order to broaden our knowledge about this area and to examine whether or not a computer can learn complicated behaviours, not only function approximations through machine learning. % Det kanske är lite skumt att säga "vi skall försöka undersöka machine learning på en djupare nivå och ta reda på detta om det går att göra dittan o dattan", skulle väl inte säga att vi kommer nå några banbrytande slutsatser direkt?

Machine learning in its simplest form, being used to solve a relatively simple problem is not very complicated. However it becomes vastly more complicated when applied to a more complex problem, where a simple rule based solution does not really suffice in order to solve the problem well enough. In order to effectively research machine learning this project will take advantage of this fact. The project will focus around the problem of taking a Formula1 (F1) car as fast as possible around any given F1 circuit, a problem that is very flexible in terms of complexity. This will allow the project to start of with the very basics behind machine learning then by slowly increasing the complexity of the problem we can closely examine the more complex concepts behind machine learning.

% Tvek på om det är värt att ha med detta stycket. Det finns redan ganska bra motiveringar till varför machine learning är vettigt och vad man kan applicera det inom i de första stycket.
% TODO Saknar ett "tryck" på slutet... någon slags avslutande mening!
The potential applications that AI and machine learning can be used within are as one might imagine endless. In this specific project, it's interesting to consider the gaming industry and racing industries for example. Could such an AI be used in racing games? Maybe even for game AIs in general? Could it potentially be of help to the F1 drivers during practice or even for the team engineers? 
% Skulle kanske kunna lära sig något intressant för racing spel, men det är inte där det fokus ligger? Skulle detta kunna vara till hjälp för generell spel ai?

\subsection{Problem}
\subsection{Other techniques}


\section{Purpose} %Copied from planning
The purpose of this project is to determine whether or not machine learning can be used in order to create an AI capable of driving a car around a F1 circuit as fast as possible in a virtually simulated environment. The result will be used to examine the possibilities and limitations of various machine learning techniques, be that whether the AI will succeed in driving the car around the circuit or not. 


\section{Limitations} %Copied from planning
There are many different ways in which one can implement the concept of AI. This project will focus on an AI that learns how to make a correct decision for a certain scenario by using some method of machine learning. This decision is based on the interest of the project participants in experimenting and learning more about the machine learning concept as well as the fact that machine learning is a modern way of thinking about AI.

When developing such an AI one needs to consider the most appropriate environment to let the AI learn and work in. The most appropriate environment, in this specific scenario will be a virtual one. A virtual environment allows for great flexibility during the AIs learning process, for example it will be possible to instantly reset the car to the tracks starting position and change both physical and environmental aspects of the world such as gravity, frictional constants on the track or width of the track.

However implementing a fully fledged physical engine into a simulation engine would require a great deal of time and dedication. In order to avoid this, one could use an already existing engine that simulates cars and their physics or simplify the problem enough that it would be reasonable both in the aspect of time and complexity to implement the physics system into a self developed simulation engine.

% TODO Gabriel from Simon: Add more physical aspects that we take into consideration in this paragraph.
In order to keep the solution as standalone as possible, making sure not to limit the possible extensions of the project in the future the decision was made to go with the second option. However this also means that the project has to be limited in terms of physical parameters at the beginning of developmen. Deciding the different aspects of the physical world that would greatly affect the final race line that the AI would eventually make the car take had to be taken into consideration. The conclusion being that a two dimensional environment that considers the current turning radius of a pre-specified car as a function of its speed will suffice. The additional complexity of implementing a simulation in a three dimensional environment would not outweigh the benefits, as the conceptual differences that the extra dimension will add for the AI is not large enough.

The goal of the AI is to consider how to approach a set of curves in order to make as fast a lap as possible. This means that the AI will not consider the scenario were more than one car is driving on the track at a time. This would raise the complexity above one that would be reasonable within the time frame of the project. Having no more than one car on the track also means that the AI will not consider any competitive racing strategy, such as when to take a pit stop or driving on a different race line in order to block a car currently trying to overtake. Furthermore, the set of curves that will be required for the AI to approach correctly are all those that were a part of the 2015 championship in Formula1.

\iffalse
Key points:

Background
- Racing problem
  - car has momentum and limited manoeuvrability
  - tactical
  - sequence of actions, done with precision
- Using machine learning with little knowledge of the domain.
- Many techniques not feasible
  - discuss

Purpose 
- The exploration of NEAT in context of the racing problem. How it may be used and what role it may have.
- Discuss what it could mean to machine learning in general.

Limitation
- Complexity of the environment
- Replicating conceptual similarities of the racing behaviour such as: reasonable positioning, maximising speed where possible, taking curves... 
- One car, not the interaction between cars in a racing setting.
\fi
