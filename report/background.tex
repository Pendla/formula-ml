% Bakgrund
% Bakgrund ska innehålla en motivering till varför det valda ämnet är intressant ur akademisk synvinkel och/eller ur tekniskt perspektiv eller i förekommande fall ur kundens/uppdragsgivarens perspektiv. I vissa fall ska den här rubriken inkludera en kort historik över ämnet. Efter att ha läst bakgrunden ska alla läsare förstå varför ämnet är relevant. Följande frågeställningar bör vara aktuella:
% Vad är ämnet/problemet som ska undersökas? Varför har ämnet/problemet uppkommit? Varför är det ett relevant eller intressant ämne/problem? För vem? Kan det specifika ämnet/problemet relateras till en mer generell diskussion?

\chapter*{Background}
Bra frågor att ställa sig:

\begin{itemize}
    \item Vad är ämnet/problemet som ska undersökas?  Creating an autonoumus formula one car using machine learning.
    \item Varför har ämnet/problemet uppkommit? Det är svårt att hitta enkla regelbaserade beteenden för många problem, eller traditionella algoritmer.
    \item Varför är det ett relevant eller intressant ämne/problem? Formel 1, autonoma fordon, machinelearning i allmänhet.
    \item För vem? Machinelearning: Möjliggör smartare program som hjälper individen mer. Racingbranchen?
    \item Kan det specifika ämnet/problemet relateras till en mer generell diskussion? Se fråga "varför är det relevant"
\end{itemize}

\begin{itemize}
  \item Motivate the purpose of the report.
  \item What is the source of the problem?
  \item Why is it interesting and/or relevant?
  \item Usually ~1 page or less.
  \item Discussion on Effect goals. Learn machine learning. this formula 1 project is a sandbox.
\end{itemize}
