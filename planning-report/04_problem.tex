Fo% Det här avsnittet är ofta den viktigaste delen av planeringsrapporten (och av den slutgiltiga rapporten). Den syftar till att identifiera frågan/frågorna som ska tas upp i projektet. Det är viktigt att gruppen gör en problem(uppgifts)analys även om det i projektförslaget redan finns ett problem (en uppgift) specificerat. Anledningen till detta är att det riktiga primära problemet ofta skiljer sig från det i början av uppdragsgivaren/förslagsställaren/kunden föreslagna. Problemanalysen syftar också till att bryta ner problemet/uppgiften i mindre och mer detaljerade delproblem/deluppgifter, vilket också leder till formulering av delsyften. Genom att göra detta får studenterna mycket bättre förståelse för de olika aspekterna av problemet/uppgiften. Utan den här informationen är det omöjligt att identifiera vilken information som behövs, vilka informationskällor som behövs, och lämpliga tillvägagångssätt.

% En bra problemanalys som identifierar delproblem/deluppgifter och delsyften vilar i många fall på användning av teorier och modeller från litteraturen. En litteraturgenomgång bör därför genomföras tidigt i processen.

\chapter{Problem Analysis}
% Introduction
% What problems and questions will be covered in the report?
% Can this problem in itself be solved by machine learning, can we achieve a complex enough behaviour.
% If we can solve the problem with machine learning, what method are applicable?
% How well can machine learning algorithms solve the problem?
In order to determine whether or not and how well machine learning can be applied to train a virtual autonomous car a number of questions need answering; Can machine learning create a complex enough behaviour? If that is the case, what methods are applicable and how well do they work? Furthermore what is the behaviour we are trying to achieve. What are the important and defining traits of a F1-drivers behaviour? 

\section{Simulating the Vehicle \& Environment}
% How do we simulate the car?
% What aspects of the real world are relevant to simulate. 
The AI that is to be developed needs to have some virtual environment in which to work. This simulated environment has to contain a F1 circuit in combination with a car that is driving on that track. This simulator raises some interesting questions. How complex need the simulator be? Can the simulated environment become to complex for the AI to solve the problem, and if so at what point does it become to complex? What aspects of the real world will affect the handling and driving of the car instead of only its performance. Should focus be put on simulating the actual car (e.g. gears, slick or wet tires, weight distribution or drag) or rather focus on the environment around the car (e.g. road condition, oxygen level in the air or the temperature outside)?

\section{Target Behaviour}
% What is the target behaviour? racing theory etc
% What decisions will the ai need to make?
The optimal behaviour varies somewhat between different types of competitions. During the qualifying lap a single driver at the time tries to get the fastest lap time. During the Grand Prix the goal is to be the first to finish or to get a good emplacement. 

In order to drive fast, a number of factors must be taking in consideration. Firstly, the way that a car accelerates, brakes and turns is by applying a force with the tyres on the road and the ability to do so is limited\cite{beckman_traction_budget}. Furthermore, turning requires a perpendicular acceleration of the car corresponding to the following formula:

\[
a_c = \frac{mv^2}{r}
\]

Where $a_c$ is the central acceleration, $m$ is the mass of the car and $r$ is the radius of the circle\cite{beckman_circular_motion}. The acceleration required do therefore increase for tighter turns and higher speeds.

When a car brakes or accelerates the weight distribution changes, leading to differences in traction capabilities. This result is a limitation of how much a car can steer, brake or accelerate at the same time\cite{beckman_weight_transfer}. Generally the fastest way to drive through a corner is to be just at the limit of traction at the near the exit\cite{beckman_racing_line_intro}.  

A racing line denotes the path a car drive around the track. The classical optimal racing line for a single right hand corner that separate two straights is to drive as far to the left, brake and turn early to take a late apex\cite{beckman_racing_line_intro}. That allows for early acceleration onto the straight that increases the average speed.

The optimal line depends both of the corner before and after, but a skilled driver might take the whole track in consideration since even small changes in the racing line can have effects for a large portions of the track \cite{beckman_racing_line_intro}. He states that it is generally the best to optimise the racing line for maximum speed on the straights. That result in that the goal of the strategy for close subsequent corners is to minimise the time for the section as a whole. There is a fine balance between optimising the time spent one corner and the position and speed to optimise the others.

The web site \textit{drivingfast.net} describe and show figures the characteristics of racing lines for a number of classical corners and how to control the car in each phase of the corner\cite{driving_fast_racing_line}\cite{driving_fast_corners}.

During a Grand Prix, when many cars compete at the same time, a driver in addition to drive fast also have to add aspects such as blocking. A driver that is chased may not drive the fastest racing line, but position itself so that the chasing car have difficulties to overtake.

\section{Modelling the Problem}
% How can we translate the problem to be compatible with a specific algorithm / knowledge model.
% How can we modify the algorithms to be compatible with our problem?
% What inputs / outputs are relevant / usable etc.
% What kind of algorithm can we use to train a knowledge model.
% What kind of knowledge models can we use, what are the requirements?
Creating the AI requires some modelling decisions. There are different kinds of knowledge models that can be used to represent an instance of the AI, and several different machine learning algorithms to train those AI instances. The different models and algorithms perform differently depending on the problem that the AI is trying to solve. Thus in order to implement a specific method, the problem needs to be translated to a format to which the method is applicable. This involves settling on a representation of the state of the system and a representation of the AI output.

% I don't like "state of the system", find another expression for this.
The state of the system include a representation of the track and car. How could this model be represented such that the different AI models can solve the problem? What is the most efficient way of representing it in terms of the final solutions performance? Should the AI output represent a set of actions or for example the control positions, i.e. the position of the steering wheel, gas pedal and brake. What information regarding the state of the model does the AI need in order to take make an appropriate decision regarding the outputs? It's also important to consider what methods of learning is applicable. Is there data available regarding how to make the perfect turn? Maybe the perfect race line for the entire circuit? If not, is it possible to encourage a certain behaviour and discourage another?

\section{Evaluating the Solution}
% How can we test/evaluate a solution.
When the AI presents a solution, there needs to be some way to evaluate this solution. Having a proper way to evaluate the result will be the only way to determine whether or not the AI has successfully found the fastest way around the track or not. In many situations the evaluation might be trivial with some basic knowledge in racing theory. However there might come situations where the evaluation becomes non-trivial. Situations where the car for example takes the most efficient racing line, but there is no to determine, way by simply looking at the simulation, if the car could have gone faster? What is the most appropriate way to evaluate these situations? Could the time spent through a part of the circuit be measured? Is it reasonable to compare this time to real world times for this part of the circuit? If not, is it possible to mathematically calculate the fastest possible time through that section?
% TODO Saknas något här.... avslut?

\iffalse
% OLD SHIT
In order to construct an artificial intelligence with the desired behaviour of a Formula 1 driver a number of questions need to be answered; Can such a complex behaviour be found through the use of machine learning? Furthermore if that is the case, what information is relevant for the AI in order to achieve the desired behaviour? How the format and size of both the input and output data will affect the outcome will also be analysed.

% Källa på hur machine learning brukar använda stora dataset.
In order to train an AI, feedback of some kind is required. Usually this is achieved by comparing the result with a large data set. Without such a set of example data an alternative method of giving feedback will be required, thus in this case a simulation will be used. This raises the question of how the simulation will affect the behaviour of the AI. Since the simulation will be limited in realism some aspects of the real world that affect the behaviour of Formula 1 drivers might get lost. Which aspects are relevant to simulate, which are not?

In order for the AI to achieve a behaviour similar to that of a Formula 1 driver it must grasp a few different concepts of varying complexity. First of all the AI must be able to handle the car and keep it on the track. This includes basic steering to avoid collision as well as braking and accelerating in and out of corners. With the ability to handle the basics the AI will have to take strategy and planning into account in order to drive close to the optimal race line. To achieve this behaviour the AI must plan a corner so that it places the car in order to maximise the speed through the corner and stay on track.

However optimising each corner locally might not lead to a global optimum. In order for the car to take a series of corners optimally it must consider several corners ahead and plan accordingly. 

\section{Racing and the physics behind it}
High end racing can be a very complicated competition with a lot of factors playing a part as to who will end up as the final victor. One does not win the competition solely on doing the fastest lap in the race. However it may play a big part in achieving a good position in the race, and in qualifying laps it is the deciding factor. In order to properly simulate a F1 car going around a circuit as fast as possible, one has to consider both the theory of racing and the physics behind it.

When a car is moving, the forces that determine in which direction the car goes are ultimately those resulting from the friction between the road and the tires. The total amount of force $F$ the tyres have the capacity to perform is limited. So in order to simulate a F1 car going around a track, one should simulate these forces and how they distribute between the different aspects of the car.

When a driver pushes the gas/brake pedal or turns the wheel a force need to be applied in order for the car to change its velocity. For example when the car accelerates, the engine pushes the car forward by applying a force by friction on the road with the tyres. When the driver turns the wheel, the position of the front tyres makes them resist motion in one direction, thus resulting in a force that accelerate the car in a different direction.

% START Jag får inte in denna sektionen någonstans... den är lite för sig själv liksom..! 
% Gabriel: Formeln som sådan är inte kanske inte värd att ha med. Man skulle behöva styrka att den är relevant. Det viktiga är konceptet att hastighet höjer svängradien
The force required to move body in a circular motion is:

\[
F_c = \frac{mv}{r}
\]

Where $F_c$ is the central force, $m$ is the mass of the body, $v$ the speed and $r$ is the radius of the circle. This simple modelling of turning show the competing relationship between distance and speed.
% END på sektionen jag inte får in någonstans

The fact that the amount of force the tyres can perform is limited results in the car not being able to turn as efficiently if the driver pushes the gas or break pedal at the same time as turning the wheel. However the amount of force the tyres are able to perform is related to the force that is pushing down on them. This force is increased by the fact that F1 cars often have wings that produces down force. The increased pressure to the ground let the tyres provide more friction. Thus resulting in a lower turning radius at a higher speed, since higher speed means more down force.

% RESPONSE Simon: Inte relevant? Är inte med i våra beräkningar..?

It is also so that the air resistant increases drastically at high speeds. In order to just keep speed, the car need exert large amount of work. This is an important factor to what the maximum speed of a car is.

How fast a driver manages to go around a corner depends on the factors above in combination with a lot more. In simple terms the lap time is the product of the average speed and the distance driven, $time = \frac{distance}{speed}$. In order to minimize this relationship one needs to find a balance between the two parameters; Due to limitations in the cars performance, it is not always beneficial to take the shortest path around a corner. Cars have a limited ability to turn (the turning radius) and this turning radius increases with the cars speed, thus in order for the car to stay on the track the speed must be limited when going around a corner. However in order to go around a corner as fast as possible, the driver may also have to increase the length of the path taken around that corner, since this will allow for a higher speed both entering and leaving the corner, thus lowering the amount of time spent through that corner.

\section{Racing lines}
Planned content:

Based on the central physical concepts involved, what is good and bad behaviour in different situations. 
\begin{itemize}
    \item What the characteristics the racing line for single bend.
    \item Explain the complexity and optimal racing line for several close bends.
    \item Recall that gas/brake limit the capacity to steer. Explain how it influences the optimal lines.
    \item Etc.
\end{itemize}
\fi